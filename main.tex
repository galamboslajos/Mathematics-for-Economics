\documentclass{article}
\usepackage{amsmath}
\usepackage{graphicx} % Required for inserting images

\title{Mathematics for Economics}
\author{Lajos Galambos}
\date{August 2024}

\begin{document}

\maketitle

\section{Introduction}
This is a note created for BA/MA level Mathematics for Economic Science based on the book of Knut Sydsæter and Peter Hammond: Essenatial Mathematics for Economic Analysis. 

\section{Algebra}
Natural Numbers – Integers – Rational Numbers – Irrational Numbers [infinite non periodic decimals] – Real Numbers

\subsection{Integer Powers}
\begin{equation}
  a^n = a \times a \times \ldots \times a \quad \text{(n times)}
\end{equation}

\begin{equation}
  a^0 = 1 \quad \text{for } a \neq 0
\end{equation}

\begin{equation}
  a^{-n} = \frac{1}{a^n}
\end{equation}

\subsection{Properties Powers}
\begin{equation}
  a^r \times a^s = a^{(r+s)}
\end{equation}

\begin{equation}
  (a^r)^s = a^{(r \times s)}
\end{equation}

\begin{equation}
  \frac{a^r}{a^s} = a^{(r - s)}
\end{equation}

\begin{equation}
  (ab)^r = a^r \times b^r
\end{equation}

\begin{equation}
  \left(\frac{a}{b}\right)^r = \frac{a^r}{b^r}
\end{equation}

\begin{equation}
  (abcde)^r = a^r \times b^r \times c^r \times d^r \times e^r
\end{equation}

\begin{equation}
  (a+b)^r \neq (a \times b)^r
\end{equation}

\subsection{Rules of Algebra}
In algebraic expressions, polynomials are defined by: \textbf{numerical coefficients}, and \textbf{terms of same type}. We can use \textbf{factoring} to make expressions more convenient to read. \\

Quadratic identities:
\begin{equation}
  (a+b)^2 = a^2 + 2ab + b^2
\end{equation}

\begin{equation}
  (a-b)^2 = a^2 - 2ab + b^2
\end{equation}

\begin{equation}
 (a+b)(a-b) = a^2 - b^2
\end{equation}

\subsection{Fractions}
Rules of fractions: 

\begin{equation}
 \frac{a\times c}{b\times c} = \frac{a}{b} \quad \text{for } b \neq 0 \text{ and } c \neq 0
\end{equation}

\begin{equation}
 \frac{-a}{-b} = \frac{a}{b} 
\end{equation}

\begin{equation}
 -\frac{a}{b} = \frac{-a}{b} 
\end{equation}

\begin{equation}
 \frac{a}{c} + \frac{b}{c} = \frac{a + b}{c} 
\end{equation}

\begin{equation}
 \frac{a}{b} + \frac{c}{d} = \frac{ad + cd}{bd} 
\end{equation} \\

LCD is a technique, finding the Least Common Denominator. It is used in especially the case of equation (18). \\

\begin{equation}
 a + \frac{b}{c} = \frac{ac + b}{c} 
\end{equation}

\begin{equation}
 a \times \frac{b}{c} = \frac{ab}{c} 
\end{equation}

\begin{equation}
 \frac{a}{b} \times \frac{c}{d} = \frac{ac}{bd} 
\end{equation}

\begin{equation}
 \frac{a}{b} \div \frac{c}{d} = \frac{a}{b} \times \frac{d}{c} 
\end{equation}

\subsection{Fractional Powers}
Powers can be written in a fractional form (if in the power we have rational numbers). Those are square roots. 

\begin{equation}
  a^{\frac{1}{2}} = \sqrt{a} \quad \text{valid if } a \geq 0
\end{equation}

\begin{equation}
 \sqrt{a} \times \sqrt{b} = \sqrt{ab}
\end{equation}

\begin{equation}
 \sqrt{\frac{a}{b}} = \frac{\sqrt{a}}{\sqrt{b}}
\end{equation}

\begin{equation}
 \sqrt{a} + \sqrt{b} \neq \sqrt{a} + \sqrt{b} \quad \text{(in general)}
\end{equation}

\begin{equation}
 a^{\frac{1}{n}} = \sqrt[n]{a} 
\end{equation}

\begin{equation}
 a^{\frac{p}{q}} = (\sqrt[q]{a})^p  \quad\text{(p an integer, q a natural number)}
\end{equation}



\end{document}